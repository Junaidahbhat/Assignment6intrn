\documentclass[a4paper,12pt]{article}

\usepackage{graphicx}
\usepackage{amsmath}
\usepackage{tkz-euclide}

\begin{document}
\title{Assignment 6}
\author{Junaid Ahmad Bhat}
\date{January 22, 2021}
\maketitle
\section*{{\small Question}}
Draw GOLD such that OL = 7.5,GL =6,GD = 6, LD = 5, OD = 10.


\section*{{\small Solution}}
\begin{center}
\begin{tikzpicture}[scale=1]
    
    \coordinate[label=left:$L$] (L) at (0,0);
    \coordinate[label=right:$D$] (D) at (5,0);
    \coordinate[label=above:$G$] (G) at (2.5,5.45);
    \coordinate[label=above:$O$] (O) at (-1.87,7.26);
  
    
    \draw (L)--node[below] {5}
    (D)--node[right] {$\textrm{6}$}
    (G)--node[above] {$\textrm{}$}
    (O)--node[left] {$\textrm{7.5}$}    
    (L)--node[below] {$\textrm{6}$}
    (G)--node[right] {$\textrm{}$}
    (D)--node[above] {$\textrm{10}$}(O);
\end{tikzpicture}

\end{center}
\hspace*{6cm} Rough Sketch\\

\textbf{TO draw the figure we need to find coordinates of all vertices}\\

Assuming L as origin i,e L(0,0)\\

Therefore coordinates of D are (5,0) \hspace*{2cm}(as LD = 5)\\

Now,let's find coordinates of vertices G and O.\\

\textbf{Coordinates of G}\\

G is the intersection point of line LG and DG.\\

Line LG:y = mx + c\\

\hspace*{0.5cm}y=tan($\angle$GLD)x+c\hspace*{2cm}(1) \\

$\angle$GLD = cos$^-$$^1$((6$^2$+5$^2$-6$^2$)/(2*6*5))\hspace*{2cm}(cosine rule)\\

Therefore,$\angle$GLD=65.37$^{\circ}$\hspace*{2cm}(2)\\

Putting eqn(2) in eqn(1),we get\\

y = 2.18x + c\\

Also c=0       \hspace{3cm} (as it passes through (0,0))\\

Therefore LG:y=2.18x  \hspace*{2cm}(A)\\

Similarly\\


Line DG:y = mx + c\\

\hspace*{0.5cm}y=tan((180$^{\circ}$-$\angle$GDL)x+c\hspace*{2cm}(3)(angles measured anti-clockwise)\\

$\angle$GLD = cos$^-$$^1$((6$^2$+5$^2$-6$^2$)/(2*6*5))\hspace*{2cm}(cosine rule)\\

Therefore,$\angle$GDL=65.37$^{\circ}$\hspace*{2cm}(4)\\

Putting eqn(4) in eqn(3),we get\\

y = -2.18x + c\\

Also c=10.9       \hspace{3cm} (as it passes through (5,0))\\

Line DG:y = -2.18x + 10.9 \hspace*{2cm}(B)\\

solving  eqn(A) and eqn(B),we get\\

\hspace*{1cm}   x = 2.5;  y = 5.45\\

Therefore coordinates of G are (2.5,5.45)\\

\textbf{Coordinates of O}\\

G is the intersection point of line LO and DO.\\

Line LO:y = mx + c\\

\hspace*{0.5cm}y=tan($\angle$OLD)x+c\hspace*{2cm}(5) \\

$\angle$OLD = cos$^-$$^1$((7.5$^2$+5$^2$-10$^2$)/(2*7.5*5))\hspace*{2cm}(cosine rule)\\

Therefore,$\angle$OLD=104.47$^{\circ}$\hspace*{2cm}(6)\\

Putting eqn(6) in eqn(5),we get\\

y = -3.87x + c\\

Also c=0       \hspace{3cm} (as it passes through (0,0))\\

Therefore LO:y= -3.87x  \hspace*{2cm}(C)\\

Similarly\\


Line DO:y = mx + c\\

\hspace*{0.5cm}y=tan((180$^{\circ}$-$\angle$ODL)x+c\hspace*{2cm}(7)(angles measured anti-clockwise) \\

$\angle$OLD = cos$^-$$^1$((10$^2$+5$^2$-7.5$^2$)/(2*10*5))\hspace*{2cm}(cosine rule)\\

Therefore,$\angle$ODL=46.56$^{\circ}$\hspace*{2cm}(8)\\

Putting eqn(8) in eqn(7),we get\\

y = -1.056x + c\\

Also c=5.28       \hspace{3cm} (as it passes through (5,0))\\

Line DO:y = -2.18x + 5.28 \hspace*{2cm}(D)\\

solving  eqn(C) and eqn(D),we get\\

\hspace*{1cm}   x = -1.87;  y = 7.26\\

Therefore coordinates of O are (-1.87,7.26)\\

















\begin{figure}[h]
\centering
\includegraphics[width=0.8\textwidth]{Assignment6intrn_2,34}
\caption{Figure using python}
\end{figure} 

\pagebreak


\section*{{\small Question}}
Draw rectangle OKAY with OK = 7 and KA =5.


\section*{{\small Solution}}

As given figure is rectangle,therefore\\

YO=5 ; AY=7       (opposite sides are equal)

\begin{center}
\begin{tikzpicture}[scale=1]
    
    \coordinate[label=left:$O$] (O) at (0,0);
    \coordinate[label=right:$K$] (K) at (7,0);
    \coordinate[label=above:$A$] (A) at (7,5);
    \coordinate[label=above:$Y$] (Y) at (0,5);
  
    
    \draw (O)--node[below] {7}
    (K)--node[right] {$\textrm{5}$}
    (A)--node[above] {$\textrm{7}$}
    (Y)--node[left] {$\textrm{5}$}    
    (O)
   ;
\end{tikzpicture}

\end{center}
\hspace*{6cm} Rough Sketch\\

\textbf{To draw triangle, we need coordinates of all vertices}

Assuming O as origin i,e O(0,0)\\

Therefore K has coordinates as (7,0)   \hspace*{2cm}(as OK=7)\\

Also it is clear;y mentioned that given figure rectangle\\

Therefore A has coordinates as (7,5)   \hspace*{1cm} (as KA=5)\\

also Y has coordinates as (0,5)  \hspace*{1cm} (Figure is rectangle)\\

\begin{figure}[h]
\centering
\includegraphics[width=0.8\textwidth]{Assignment6intrn_2,41}
\caption{Figure using python}
\end{figure}

\end{document}
